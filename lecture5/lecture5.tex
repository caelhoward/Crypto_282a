\documentclass[11pt]{article}

% Prefix for numedquestion's
\newcommand{\questiontype}{Question}


% Use this if your "written" questions are all under one section
% For example, if the homework handout has Section 5: Written Questions
% and all questions are 5.1, 5.2, 5.3, etc. set this to 5
% Use for 0 no prefix. Redefine as needed per-question.
\newcommand{\writtensection}{0}

\usepackage{amsmath, amsfonts, amsthm, amssymb}  % Some math symbols
\usepackage{mathtools}

\usepackage{centernot}
\usepackage{mathtools}

\setlength{\parindent}{0pt}

\begin{document}

\textbf{Computational Zero-Knowledge:}
$$S^{(V^*)}(x) \overset{\sim}{=} [P \leftrightarrow V(x)]$$
View of the two are the same as msg of $P$ and $V^*$ with $V^*$'s randomness.

\section{Commitment Protocol}
\textbf{Commitment Protocol}: An interactive protocol that consists of the commitment phase and open phase satisfying two properties:
\begin{itemize}
    \item Binding: After the commit protocol terminiates, the commiter can only open $b$ in 1 way.
    \item Hiding: After the commit phases is over, any PPT receiver cannot predict bit better than $\frac{1}{2} + \text{negl}$
\end{itemize}

\section{Graph 3-Coloring}
Given a graph $G$, we say it is \textbf{3-colorable} if we can assign colors to each node such that for all $(u,v) \in E$, $u$ and $v$ are different colors.
\vspace{1em}

This problem is NP-complete. The problem of Circuit satisfiability is NP-Complete. Using 3-coloring, we can create gadgets that represent Or, And, and Not gates. Since these gates are turing complete, we can convert any circuit into a corresponding graph, such that one color represents $0$, another represents $1$, and the third color is used to assist with the gates. This way we know if the graph is 3-colorable, then the circuit it represents is satisfiable.

\subsection{ZK Proof for 3-Color}
Given a graph $G$, we have the following protocol to demonstrate that $G \in $ 3-Color:
\begin{enumerate}
    \item The prover $P$ creates a random permutation shuffling the colors from one to another randomly. (exp. Red $\rightarrow$ Blue). The prover secretly commits to this new three coloring and sends commitments to $V$.
    \item $V$ selects a random edge $e = (u,v)$ and sends their selection back to $P$.
    \item $P$ decommits nodes $u$ and $v$ to $V$ who can see that they are different colors.
    \item This process is repeated $k$ times.
\end{enumerate}
\textbf{Completeness:} is clear from the inspection.

\textbf{Soundness:} Consider some $G \notin $ \textbf{3-Color}. Then, $\exists e = (u,v) \in E$ where $(u,v)$ are the same color. Since $V$ selects a random edge $e'$, the probability that they choose this edge is $\frac{1}{|E|}$. Therefore the probability that $P$ can successfully fool the verifier is:
\begin{align*}
    (1-\frac{1}{|E|})^k
\end{align*}
Which for $k = |E|$ is $1/e^{|E|}$.

\subsection{Simulator for Protocol}
To demonstrate this protocol is Zero Knowledge, we will create simulator $S$ which does the following:
\begin{enumerate}
    \item The simulator will pick a random edge $e$ and commit two different colors to the edge's nodes.
    \item If the verifier asks to reveal any edge that is not $e$, rewind the tape and try again.
    \item Repeat until verifier accepts, since each attempt has a $1/|E|$ chance of succeeding, it will take approximately $k \cdot |E|$ tries to succeed. 
\end{enumerate}

\section{Blum's Protocol for Hamiltonian Cycle}
\textbf{Hamiltonian Cycle} is a decision problem that asks if there exists a simple cycle of length $n$ inside of graph $G$. This problem is NP-Complete as one can map an instance of $3$-Sat to an instance of this problem. The following protocol is a ZK proof for verifying if $G \in $ Ham-Cylce:
\begin{enumerate}
    \item Prover $P$ picks a random permutation $\pi$ and commits an adjacency matrix of $\pi(G)$ to $V$.
    \item $V$ selects a random $b \overset{\$}{\leftarrow} \{0,1\}$ and sends $b$ to $P$.
    \item If $P$ obtains $0$, it opens the entire adjacency matrix and sends it to $V$, along with the permutation of $\pi^{-1}$ that maps the adjacency matrix it received back to $G$. If $P$ obtains $1$, it decommits the permuted cycle to $V$.
    \item $V$ checks $P$'s decommitments and verifies that if $b=0$, it properly permutes back to $G$, and if $b=1$, $P$ correctly decommited a single $1$ in every row and column.
\end{enumerate}
\textbf{Notes on Protocol}:

An adjacency matrix is a 0/1 matrix that represents an outgoing edge from vertex $i$ to vertex $j$, which would result in something like $adj[i][j] = 1$. Since a hamiltonian cycle visits each vertex in the graph exactly once other than the starting vertex, there are $n$ edges that exist, each starting at a unique source vertex and ending at a unique destination vertex. Therefore, if there exists a hamiltonian cycle in a graph, the adjaceny matrix can illustrate this by presenting $n$ entries of $1$'s such that each is the only 1 in its column and row.
\vspace{1em}

\textbf{Completeness:} In the case that $V$ returns $0$, it is clear that $P$ can decommit each entry and also send $V$ $\pi^{-1}$. In the case that $V$ sends $1$, since there exists a Hamiltonian Path in $G$, there must also exists one in $\pi(G)$. Therefore, there must be some combination of $n$ 1's such that each $1$ is unique in its row and column. This is what $P$ decommits to demonstrate there must have been a cycle in the original $G$.
\vspace{1em}

\textbf{Soundness:} Consider a graph $G$ that does not have a Hamiltonian cycle. This means that you cannot find $n$ entries in the matrix that demonstrate a hamiltonian cycle. Therefore, whatever commitment you send to $V$ can be either:
\begin{enumerate}
    \item A permutation of $G$ without a hamiltonian cycle
    \item A adjacency matrix with a hamiltonian cycle that is not a permutation of $G$.
\end{enumerate}
Therefore, $P$ can only correctly respond to at most $1$ $b$ request from $V$, meaning that the probability it passes $k$ rounds of this protocol is $1/2^k$.

\section{Commitment Protocols}
One Way Functions: Functions that are easy to compute and hard to invert.
$$x \overset{\$}{\leftarrow} X, f(x) = y \text{ is easy.}$$
$$\forall y, \text{ calculating } f^{-1}(y) \text{ is hard}$$

For example, we consider factoring to be a hard problem, so given two large primes $p,q$, calculating $N = p \cdot q$ is easy, but finding $p,q$ given $N$ is hard.

\subsection{Implications of Existence}
We have the following relation where each statement is implied by all following statements.
\begin{enumerate}
    \item $P \neq NP$, or in other words, there exists some hard problems.
    \item Average $P \neq$ Average $NP$, or there exists hard problems that are easy to sample.
    \item $\exists$ One Way Functions 
\end{enumerate}
\subsection{Definition of One Way Function}
A one way function can be loosely defined as follows. There exists a challenger and adversary such that if the challenge sends the adversary $y=f(x)$ for a random $x \leftarrow X$, then the adversary wins if it can find an $x'$ such that $y = f(x')$. We say a function $f$ is one-way if the adversaries chances of winning quickly approaches 0 as $|x| = k$ increases.

\subsubsection{Formal Definition}
A function $f$ is one way if
$$\forall k, \forall A^{*} \in \text{PPT}, \forall c, \text{Pr}[A^*(f(x),1^n) \in f^{-1}(f(x))] < \frac{1}{k^c}$$
This is saying, for all values of $k$ (length of x), and for all adversaries $A^*$, the probability that $A^*$ can find some $x'$ where $f(x') = f(x)$ descreases faster than the inverse of any polynomial of $k$.

\end{document}