\documentclass[11pt]{article}

% Prefix for numedquestion's
\newcommand{\questiontype}{Question}


% Use this if your "written" questions are all under one section
% For example, if the homework handout has Section 5: Written Questions
% and all questions are 5.1, 5.2, 5.3, etc. set this to 5
% Use for 0 no prefix. Redefine as needed per-question.
\newcommand{\writtensection}{0}

\usepackage{amsmath, amsfonts, amsthm, amssymb}  % Some math symbols
\usepackage{mathtools}

\usepackage{centernot}
\usepackage{mathtools}

\setlength{\parindent}{0pt}

\begin{document}

\textbf{Problem 1:} Your friend shows you an algorithm INV that can invert a given function f in PPT, but only for those f(x)
where the first 20 high-order bits of x are all 1’s. However, if the first 20 bits are not all 1’s, you assume
that for any PPT adversary A, the probability that A inverts F becomes negligible as the length of the
input increases. Could f be a one-way function? (Prove your answer). If not, is there a way to use f as a
building block to build a one-way function? If so, show how to do it and prove your answer; if not, show
a counterexample.
\vspace{1em}

\textbf{Ans:} The function is not one way. This is because, for a constant fraction $(1/2^{20})$, we can invert $f$ in PPT. We can use $f$ as a building block for a one way function in the following manner by constructing function $f'$. On input $x$, we can return $f(x) \oplus f(\hat{x})$. Only one of the two outputs can have the first 20 bits as $1$, so
\vspace{1em}

\textbf{Problem 6.} Consider a secure pseudo-random generator (PRG) G that maps n-bit seed s to 2n-bit output
G(s). Define LH(*) to be a function that takes as an input 2n bits, and outputs the first half of its input
bits. Define RH(*) similarly (for the right half). Define:
$$F(s) = G(LH(G(s)))G(RH(G(s)))$$
\begin{enumerate}
    \item The output length is $4n$.
    \item We demonstrate the security of $G$ by reduction, showing that an adversary that can distinguish a random number from $F(s)$ can differentiate a random number from $G(s)$ with non-negligible probability. This is done using the hybrid argument. We demonstrate that if we have an output that can differentiate between an output of $4$ random $n$-bit long numbers and $G(s)$, $G$ is not a secure PRG. Firstly, this implies the existence of an adversary $A$ that outputs $1$ if the input is $G(s)$ with non-negligible probability. The idea is the following. First, we replace the left $n$ bits of $G(s)$ with a truly random number. Therefore, the final output of this hybrid machine will be:
    $$F_1(S) = G(LH(r_l))G(RH(G(s)))$$
    \begin{enumerate}
        \item Firs
    \end{enumerate}
\end{enumerate}
\end{document}